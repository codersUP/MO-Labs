\documentclass[titlepage, 11pt]{scrartcl}
\usepackage{graphicx}
\usepackage[utf8]{inputenc}
\usepackage{hyperref}
\usepackage{amsmath, amsthm, amsfonts}
\usepackage{graphics}
\usepackage{skull}
\usepackage{mathabx}
\usepackage{float}
\usepackage{epsfig}
\usepackage{amssymb}
\usepackage{dsfont}
\usepackage{latexsym}
\usepackage{newlfont}
\usepackage{epstopdf}
\usepackage{amsthm}
\usepackage{epsfig}
\usepackage{caption}
\usepackage{multirow}
\usepackage{graphics}
\usepackage{wrapfig}
\usepackage[rflt]{floatflt}
\usepackage{multicol}
\usepackage{minted}

\hypersetup{colorlinks,%
	citecolor=black,%
	filecolor=cyan,%
	linkcolor=blue!60,%
	urlcolor=cyan}

\title{
    \normalfont\normalsize
    {\huge Modelos de Optimización\\
    		\textbf{Laboratorio 7:} Uso de software para solución numérica de problemas de 
    		Optimización Lineal Discreta}
    \vspace{12pt}
}

\author{Osmany P\'erez Rodr\'iguez\\
		Enrique Mart\'inez Gonz\'alez\\
		Carmen Irene Cabrera Rodr\'iguez\\
		\textbf{Grupo C412}}
	

\date{}

\begin{document}
    \maketitle 
    
    \section{Python - MIP}
    	Los paquetes de Python-MIP constituyen una herramienta para el modelado y solución de Problemas de Programación Lineal Enteros Mixtos (\textit{MIPs - Mixed-Integer Linear Programming Problems}) en Python. Contiene diferentes solucionadores para realizar la optimización, entre ellos: \href{https://github.com/coin-or/Cbc}{CBC} (\textit{COIN-OR Branch-and-Cut}), altamente configurable, que fue el utilizado para hallar la solución de los ejercicios propuestos.
		Se empleó la clase \textbf{Model}, que permite la declaración de las variables, las restricciones y la función objetivo de manera muy sencilla y cómoda. El método \textbf{optimize} es el que, finalmente computa el algoritmo que resuelve el sistema y permite, de forma opcional, setear el máximo tiempo de ejecución del algoritmo, la máxima cantidad de soluciones a encontrar, entre otros. Este método retorna el resultado del cómputo a través del \textit{status}, que puede ser:
		\begin{itemize}
			\item OPTIMAL: La búsqueda fue concluida y una solución óptima fue encontrada.
			\item FEASIBLE: Existe una solución factible que fue encontrado, pero no hubo tiempo de probar su optimalidad.
			\item NO\_SOLUTION\_FOUND: No se encontró ninguna solución.
			\item INFEASIBLE: No existe solución factible para el modelo.
			\item INT\_INFEASIBLE: La relajación del problema lineal tiene solución factible pero no existe solución entera que sea factible.
			\item UNBOUNDED: Una o más variables que aparecen en la función objetivo no están incluidas en las restricciones vinculantes y el valor objetivo óptimo es infinito.
			\item ERROR: Si ocurrió algún error durante la optimización.
		\end{itemize}
	
		Puede encontrar más información acerca de esta biblioteca a través del siguiente \href{https://docs.python-mip.com/en/latest/index.html}{enlace}.
		
	\section{Solución de los problemas propuestos}
	\subsection{Ejercicio 1}
	 El algoritmo empleado fue el siguiente:
	\begin{figure}[H]
		\begin{minted}[tabsize=4]{python}
def solve():
	m = Model(sense=MAXIMIZE, solver_name=CBC)
	
	# Variables
	x1 = m.add_var(var_type=INTEGER, lb=0)
	x2 = m.add_var(var_type=INTEGER, lb=0)
	
	# Constraints
	m += x1 + 3 * x2 <= 5
	m += 2 * x1 + x2 <= 6
	
	# Objective function
	m.objective = -2 * x1 + 3 * x2
	
	m.optimize()
		\end{minted}
		\caption{Código completo \href{ex1.py}{aquí}}.
	\end{figure}

		 Los resultados obtenidos fueron:
	\begin{figure}[H]
		\begin{minted}[tabsize=4]{python}
# OptimizationStatus.OPTIMAL
# Solution to the objective function: 3.0
# x1: 0.0
# x2: 1.0
# Execution time: 0.012865543365478516
		\end{minted}
	\end{figure}

La solución encontrada coincide con la reportada en la solución de la clase práctica, ya sea por el método Primal Todo Entero o el de Gomory I. El valor de la función objetivo se devuelve con el signo opuesto en la CP puesto que maximizar la función objetivo es equivalente a hallar el mínimo de esta función multiplicado por $-1$.

	\subsection{Ejercicio 2}
El algoritmo empleado fue el siguiente:
\begin{figure}[H]
	\begin{minted}[tabsize=4]{python}
def solve():
	m = Model(solver_name=CBC)
	
	# Variables
	x1 = m.add_var(var_type=INTEGER, lb=0)
	x2 = m.add_var(var_type=INTEGER, lb=0)
	
	# Constraints
	m += 2 * x1 + 2 * x2 == 3
	m += x1 + 2 * x2 <= 2
	
	# Objective function
	m.objective = x1 + 2 * x2
	
	m.optimize()
	\end{minted}
	\caption{Código completo \href{ex2.py}{aquí}}.
\end{figure}

Los resultados obtenidos fueron:
\begin{figure}[H]
	\begin{minted}[tabsize=4]{python}
	# OptimizationStatus.INFEASIBLE
	# Execution time: 0.011461734771728516
	\end{minted}
\end{figure}

En la CP se puede apreciar que tanto al aplicar el método Primal Todo Entero, como el de Gomory I se determina que no existe solución factible del problema. De igual forma, el estatus que retorna el método optimize: INFEASIBLE implica que el problema no tiene solución factible. Note que al tratarse de variables enteras la primera restricción del problema no tiene sentido; por tanto era de esperar este resultado.

	\subsection{Ejercicio 3}
El algoritmo empleado fue el siguiente:
\begin{figure}[H]
	\begin{minted}[tabsize=4]{python}
def solve():
	m = Model(solver_name=CBC)
	
	# Variables
	x1 = m.add_var(var_type=INTEGER, lb=0)
	x2 = m.add_var(var_type=INTEGER, lb=0)
	
	# Constraints
	m += 2 * x1 + 3 * x2 == 6
	m += 2 * x1 + 9 * x2 <= 6
	
	# Objective function
	m.objective = x1 + x2
	
	m.optimize()
	
	\end{minted}
	\caption{Código completo \href{ex3.py}{aquí}}.
\end{figure}

Los resultados obtenidos fueron:
\begin{figure}[H]
	\begin{minted}[tabsize=4]{python}
	# OptimizationStatus.OPTIMAL
	# Solution to the objective function: 3.0
	# x1: 3.0
	# x2: 0.0
	# Execution time: 0.013741016387939453
	\end{minted}
\end{figure}

La solución obtenida coincide con la reportada en la clase práctica.

	\subsection{Ejercicio 4}
El algoritmo empleado fue el siguiente:
\begin{figure}[H]
	\begin{minted}[tabsize=4]{python}
def solve():
	m = Model(sense=MAXIMIZE, solver_name=CBC)
	
	# Variables
	x1 = m.add_var(var_type=INTEGER, lb=0)
	x2 = m.add_var(var_type=INTEGER, lb=0)
	x3 = m.add_var(var_type=INTEGER, lb=0)
	
	# Constraints
	m += 3 * x1 + 2 * x2 <= 10
	m += x1 + 4 * x2 <= 11
	m += 3 * x1 + 3 * x2 + x3 == 13
	
	# Objective function
	m.objective = 4 * x1 + 5 * x2 + x3
	
	m.optimize()
	
	\end{minted}
	\caption{Código completo \href{ex4.py}{aquí}}.
\end{figure}

Los resultados obtenidos fueron:
\begin{figure}[H]
	\begin{minted}[tabsize=4]{python}
	# OptimizationStatus.OPTIMAL
	# Solution to the objective function: 19.0
	# x1: 2.0
	# x2: 2.0
	# x3: 1.0
	# Execution time: 0.015565633773803711
	\end{minted}
\end{figure}
La solución obtenida coincide con la reportada en la clase práctica.

\subsection{Ejercicio 6}
El algoritmo empleado fue el siguiente:
\begin{figure}[H]
	\begin{minted}[tabsize=4]{python}
def solve():
	m = Model(sense=MAXIMIZE, solver_name=CBC)
	
	# Variables
	x1 = m.add_var(var_type=INTEGER, lb=0, ub=4)
	x2 = m.add_var(var_type=INTEGER, lb=0)
	
	# Constraints
	m += 2 * x1 + x2 <= 10
	m += -x1 + x2 <= 5
	
	# Objective function
	m.objective = x1 + 2 * x2
	
	m.optimize()
	\end{minted}
	\caption{Código completo \href{ex6.py}{aquí}}.
\end{figure}

Los resultados obtenidos fueron:
\begin{figure}[H]
	\begin{minted}[tabsize=4]{python}
# OptimizationStatus.OPTIMAL
# Solution to the objective function: 14.0
# x1: 2.0
# x2: 6.0
# Execution time: 0.015373945236206055
	\end{minted}
\end{figure}
La solución obtenida coincide con la reportada en la clase práctica.

\subsection{Ejercicio 7}
El algoritmo empleado fue el siguiente:
\begin{figure}[H]
	\begin{minted}[tabsize=4]{python}
def solve():
	m = Model(solver_name=CBC)
	
	# Variables
	x1 = m.add_var(var_type=INTEGER, lb=0)
	x2 = m.add_var(var_type=INTEGER, lb=0)
	x3 = m.add_var(var_type=INTEGER, lb=0)
	
	# Constraints
	m += 2 * x1 - 3 * x2  + 3 * x3 <= 4
	m += 4 * x1 + x2  + x3 <= 8
	
	# Objective function
	m.objective = -2 * x1 - x2 - x3
	
	m.optimize()
	
	\end{minted}
	\caption{Código completo \href{ex7.py}{aquí}}.
\end{figure}

Los resultados obtenidos fueron:
\begin{figure}[H]
	\begin{minted}[tabsize=4]{python}
	# OptimizationStatus.OPTIMAL
	# Solution to the objective function: -8.0
	# x1: 0.0
	# x2: 8.0
	# x3: 0.0
	# Execution time: 0.5075554847717285
	\end{minted}
\end{figure}
El óptimo de la función objetivo, el valor $-8$, coincide con el obtenido en la solución reportada en la clase práctica; sin embargo, en aquel caso, este óptimo se alcanza en el punto $(0, 4, 4)$, mientras que el punto que resulta del algoritmo es $(0, 8, 0)$. Como ambos puntos pertenecen al conjunto de soluciones factibles, y se alcanza el mismo óptimo en los dos casos; se puede concluir que ambas son soluciones óptimas.

\end{document}