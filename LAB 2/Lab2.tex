\documentclass[titlepage, 11pt]{scrartcl}
\usepackage{graphicx}
\usepackage[utf8]{inputenc}
\usepackage{hyperref}
\usepackage{amsmath, amsthm, amsfonts}
\usepackage{graphics}
\usepackage{skull}
\usepackage{mathabx}
\usepackage{float}
\usepackage{epsfig}
\usepackage{amssymb}
\usepackage{dsfont}
\usepackage{latexsym}
\usepackage{newlfont}
\usepackage{epstopdf}
\usepackage{amsthm}
\usepackage{epsfig}
\usepackage{caption}
\usepackage{multirow}
\usepackage{graphics}
\usepackage{wrapfig}
\usepackage[rflt]{floatflt}
\usepackage{multicol}

\hypersetup{colorlinks,%
	citecolor=black,%
	filecolor=cyan,%
	linkcolor=blue!60,%
	urlcolor=cyan}

\title{
    \normalfont\normalsize
    {\huge Modelos de Optimización\\
    		\textbf{Laboratorio 2}}
    \vspace{12pt}
}

\author{Osmany P\'erez Rodr\'iguez\\
		Enrique Mart\'inez Gonz\'alez\\
		Carmen Irene Cabrera Rodr\'iguez\\
		\textbf{Grupo C412}}

\date{}

\begin{document}
    \maketitle 
    
	La programaci\'on de los ejercicios orientados se realiz\'o en todos los casos en el lenguaje de programaci\'on Python. La ejecuci\'on de los mismos fue llevada a cabo en una computadora HP con procesador AMD10 y 8GB RAM.
    \section{}{
		Los siguientes ejercicios (ej 1, 2, 4 de la CP2) fueron resueltos haciendo uso de \textbf{\textbf{PuLP}}, una biblioteca de Python para optimizaci\'on lineal.
    	\begin{description}
    		\item[1] 
    		
    		\textbf{Parámetros}
    			
   				$CP_i$: Cantidad de personas en la ruta $i$.
   				
   				$CG_i$: Cantidad de \'omnibus en la ruta $i$.
   				
   				$T$: Total de ómnibus nuevos.
    		
    		
    		\textbf{Variables:}
   				$X_i$: Cuántos ómnibus nuevos irán por la ruta $i$.
   				
   				$Y_i$: Es 1 si se cambian todos los ómnibus de la ruta $i$. 0 en otro caso.
    		
    		\textbf{Modelo:}
    			$$max \sum_{i = 1}^{3}ParteEnteraPorDebajo(CP_i/CG_i) * X_i$$
    			$$max \sum_{i = 1}^{3}CP_i * Y_i$$
    	
    		
    		\textbf{Restricciones:}
    			$$\sum_{i=1}^{3}X_i \leq T$$
    			$$0 \leq X_i \leq CG_i, \forall i \in \{1, 2, 3\}$$
    			
    			$$\sum_{i = 1}^{3}CG_i * Y_i \leq T$$
    		
    	
    		El c\'odigo empleado para su implementaci\'on se puede encontrar en este \href{lab2ex1.py}{enlace}. Los resultados obtenidos fueron los siguientes:
    		
    		
    		\begin{align*}
    			Cambio \ de \ parte \ de \ los \ omnibus \ ruta_1 &= 2.0\\
    			Cambio \ de \ parte \ de \ los \ omnibus \ ruta_2 &= 17.0\\
    			Cambio \ de \ parte \ de \ los \ omnibus \ ruta_3 &= 16.0\\
    			Se \ moveran \ en \ omnibus \ nuevos \ unicamente \ &37701.0 \ pasajeros
    		\end{align*}

			\begin{align*}
				Cambio \ de \ todos \ los \ omnibus \ ruta_1 &= 0.0 \\
				Cambio \ de \ todos \ los \ omnibus \ ruta_2 &= 1.0 \\
				Cambio \ de \ todos \ los \ omnibus \ ruta_3 &= 1.0 \\
    			Se \ moveran \ en \ omnibus \ nuevos \ unicamente &\ 36000.0 \  pasajeros
			\end{align*}    		
    		
    		En ambos casos la cantidad de iteraciones es 0 y el tiempo de soluci\'on es 0.00
    		
    		\item[2] 
    		
    		\textbf{Parámetros}

    				$E$: Costo de encendido de la planta.
    				
    				$LP_i$: Litros que consume un kilogramo de producción del producto $i$.
    				
    				$PP_i$: Costo de producir un kilogramo del producto $i$.
    				
    				$P_i$: Precio de venta de un kilogramo del producto $i$.
    				
    				$CL$: Cantidad de leche total.
    				
    				$PT$: Presupuesto total.
    		
    		\textbf{Variables:}
    				$X_i$: Cantidad de kilogramos que se producen del producto $i$.
    				
    				$Y_i$: Es 1 si se enciende la planta de producción del producto $i$. 0 en otro caso.

    		
    		\textbf{Modelo:}
    			$$max \sum_{i=1}^{2}X_i * P_i - X_i * PP_i - Y_i * E$$
    		
    		
    		\textbf{Restricciones:}
    			$$\sum_{i=1}^{2}X_i*LP_i \leq CL$$
    			$$\sum_{i=1}^{2}X_i * PP_i + Y_i * E \leq PT$$
    			$$0 \leq X_i \leq \frac{CL}{LP_i} * Y_i, \forall i \in \{1, 2\}$$
    		
    		
			El c\'odigo empleado para su implementaci\'on se puede encontrar en este \href{lab2ex1.py}{enlace}. Los resultados obtenidos fueron los siguientes:
			
			\begin{align*}
				Encendido de las fabricas:& Gouda \rightarrow 1\\
				& Edam \rightarrow 0\\
				Produccion (kg):& Gouda \rightarrow 800\\
				& Edam \rightarrow 0	\\
				Ganancias: & 1900		
			\end{align*}
			
			Total de iteraciones = 0
			
			Tiempo = 0.00
			
			\item[4]
			
			\textbf{Parámetros}
					$A_{d, c}$: Cantidad de estudiantes que no podrán asistir el día $d$ a la asignatura $c$.

			
			\textbf{Variables:}
					$X_{d, c}$: Es 1 si se impartirá el día $d$ la asigantura $c$. 0 en otro caso.
			
			\textbf{Modelo:}
				$$min \sum_{d=1}^{5} \sum_{c=1}^{5}X_{d, c} * A_{d, c}$$
			
			
			\textbf{Restricciones:}
				$$\sum_{d=1}^{5}X_{d, c} = 1, \forall c \in \{1, 2, 3, 4, 5\}$$
				$$\sum_{c=1}^{5}X_{d, c} = 1, \forall d \in \{1, 2, 3, 4, 5\}$$
			
			Los resultados obtenidos son los siguientes:
			\begin{align*}
				Seleccionado\_jueves\_1 &= 0.0\\
				Seleccionado\_jueves\_2 &= 0.0\\
				Seleccionado\_jueves\_3 &= 1.0\\
				Seleccionado\_jueves\_4 &= 0.0\\
				Seleccionado\_jueves\_5 &= 0.0\\
				Seleccionado\_lunes\_1 &= 0.0\\
				Seleccionado\_lunes\_2 &= 1.0\\
				Seleccionado\_lunes\_3 &= 0.0\\
				Seleccionado\_lunes\_4 &= 0.0\\
				Seleccionado\_lunes\_5 &= 0.0\\
				Seleccionado\_martes\_1 &= 1.0\\
				Seleccionado\_martes\_2 &= 0.0\\
				Seleccionado\_martes\_3 &= 0.0\\
				Seleccionado\_martes\_4 &= 0.0\\
				Seleccionado\_martes\_5 &= 0.0\\
				Seleccionado\_miercoles\_1 &= 0.0\\
				Seleccionado\_miercoles\_2 &= 0.0\\
				Seleccionado\_miercoles\_3 &= 0.0\\
				Seleccionado\_miercoles\_4 &= 0.0\\
				Seleccionado\_miercoles\_5 &= 1.0\\
				Seleccionado\_viernes\_1 &= 0.0\\
				Seleccionado\_viernes\_2 &= 0.0\\
				Seleccionado\_viernes\_3 &= 0.0\\
				Seleccionado\_viernes\_4 &= 1.0\\
				Seleccionado\_viernes\_5 &= 0.0\\
				Perdida \ de \ asistencias \ total&=  7.0			
			\end{align*}
			
			Total de iteraciones = 0
			
			Tiempo = 0.00
    	\end{description}

	}

	\section{}{
		Para la soluci\'on de este ejercicio (cada inciso) se utiliz\'o la biblioteca \textbf{Gekko}, un paquete de Python para \textit{machine learning} y optimizaci\'on. Adem\'as viene aparejada de solucionadores a gran escala para programaci\'on lineal, no lineal, cuadr\'atica, etc.
		
		
		\begin{description}
			\item[a] El c\'odigo empleado para su soluci\'on se puede encontrar a trav\'es de este \href{lab2ex2a.py}{enlace}. Los resultados obtenidos fueron los siguientes:
				\begin{align*}
					x_0 &= 16.0\\
					x_1 &= 0.0\\
					Objetivo &: -256.0 
				\end{align*}
			Cantidad de iteraciones realizadas: 3
			
			Tiempo de soluci\'on: 0.111300000004121.
			
			\item[b] El c\'odigo empleado para su soluci\'on se puede encontrar a trav\'es de este \href{lab2ex2b.py}{enlace}. Los resultados obtenidos fueron los siguientes:
			\begin{align*}
			x_0 &= 0.0\\
			x_1 &= 0.0\\
			Objetivo &: 1.0 
			\end{align*}
			Cantidad de iteraciones realizadas: 4
			
			Tiempo de soluci\'on: 3.280000000086147E-002.
			
			\item[c] El c\'odigo empleado para su soluci\'on se puede encontrar a trav\'es de este \href{lab2ex2c.py}{enlace}. Los resultados obtenidos fueron los siguientes:
			\begin{align*}
			x_0 &= 5.0\\
			x_1 &= 0.0\\
			Objetivo &: -125.0 
			\end{align*}
			Cantidad de iteraciones realizadas: 3
			
			Tiempo de soluci\'on: 2.679999999236315E-002.
		\end{description}
	
	}
        

\end{document}