\documentclass[titlepage, 11pt]{scrartcl}
\usepackage{graphicx}
\usepackage[utf8]{inputenc}
\usepackage{hyperref}
\usepackage{amsmath, amsthm, amsfonts}
\usepackage{graphics}
\usepackage{skull}
\usepackage{mathabx}
\usepackage{float}
\usepackage{epsfig}
\usepackage{amssymb}
\usepackage{dsfont}
\usepackage{latexsym}
\usepackage{newlfont}
\usepackage{epstopdf}
\usepackage{amsthm}
\usepackage{epsfig}
\usepackage{caption}
\usepackage{multirow}
\usepackage{graphics}
\usepackage{wrapfig}
\usepackage[rflt]{floatflt}
\usepackage{multicol}

\hypersetup{colorlinks,%
	citecolor=black,%
	filecolor=cyan,%
	linkcolor=blue!60,%
	urlcolor=cyan}

\title{
    \normalfont\normalsize
    {\huge Modelos de Optimización\\
    		\textbf{Laboratorio 3}}
    \vspace{12pt}
}

\author{Osmany P\'erez Rodr\'iguez\\
		Enrique Mart\'inez Gonz\'alez\\
		Carmen Irene Cabrera Rodr\'iguez\\
		\textbf{Grupo C412}}

\date{}

\begin{document}
    \maketitle 
    
	\textbf{Condiciones necesarias de Karush-Kuhn-Tucker (KKT) para problemas con restricciones de igualdad y desigualdad.}
	
	Sea $X$ un conjunto abierto no vacío en $R^n$ y sean $f$, $g_j$ y $h_i$ con $i = 1 \ldots m$ y $j = 1\ldots k$, funciones de $R^n$ en $R$. Considere el problema $P: min{f(x) | x \in X, g_j(x) \leq 0, h_i(x) = 0}$. Sea $x^*$ una solución factible del problema $P$. Suponga además que en el punto $x^*$ las funciones $f$, $h_i , \ i = 1 \ldots m$ y $g_j, \ j \in I(x^*)$ son continuamente diferenciables y la función $g_j, \ j\notin I(x^*)$ es continua.
	
	Si $x^*$ es un punto regular y un mínimo local del problema $P$, entonces existen escalares únicos $\lambda_i$

\end{document}