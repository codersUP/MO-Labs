\documentclass[titlepage, 11pt]{scrartcl}
\usepackage{graphicx}
\usepackage[utf8]{inputenc}
\usepackage{hyperref}
\usepackage{amsmath, amsthm, amsfonts}
\usepackage{graphics}
\usepackage{skull}
\usepackage{mathabx}
\usepackage{float}
\usepackage{epsfig}
\usepackage{amssymb}
\usepackage{dsfont}
\usepackage{latexsym}
\usepackage{newlfont}
\usepackage{epstopdf}
\usepackage{amsthm}
\usepackage{epsfig}
\usepackage{caption}
\usepackage{multirow}
\usepackage{graphics}
\usepackage{wrapfig}
\usepackage[rflt]{floatflt}
\usepackage{multicol}
\usepackage{minted}

\hypersetup{colorlinks,%
	citecolor=black,%
	filecolor=cyan,%
	linkcolor=blue!60,%
	urlcolor=cyan}

\title{
    \normalfont\normalsize
    {\huge Modelos de Optimización\\
    		\textbf{Laboratorio 3}}
    \vspace{12pt}
}

\author{Osmany P\'erez Rodr\'iguez\\
		Enrique Mart\'inez Gonz\'alez\\
		Carmen Irene Cabrera Rodr\'iguez\\
		\textbf{Grupo C412}}

\date{}

\begin{document}
    \maketitle 
    
	\textbf{Condiciones necesarias de Karush-Kuhn-Tucker (KKT) para problemas con restricciones de igualdad y desigualdad.}
	
	Sea $X$ un conjunto abierto no vacío en $R^n$ y sean $f$, $g_j$ y $h_i$ con $i = 1 \ldots m$ y $j = 1\ldots k$, funciones de $R^n$ en $R$. Considere el problema $P: min{f(x) | x \in X, g_j(x) \leq 0, h_i(x) = 0}$. Sea $x^*$ una solución factible del problema $P$. Suponga además que en el punto $x^*$ las funciones $f$, $h_i , \ i = 1 \ldots m$ y $g_j, \ j \in I(x^*)$ son continuamente diferenciables y la función $g_j, \ j\notin I(x^*)$ es continua.
	
	Si $x^*$ es un punto regular y un mínimo local del problema $P$, entonces existen escalares únicos $\lambda_i, \ i = 1 \ldots m$ y $\mu_j, \ j \in I(x^*)$, tal que:
	\begin{equation*}
		\nabla f(x^*) + \sum_{i = 1}^{m}\lambda_i \nabla h_i(x^*) + \sum_{j \in I(x^*)}\mu_j \nabla g_j(x^*) = 0
	\end{equation*}
	\begin{equation*}
		\mu_j \geq 0, \ j \in I(x^*)
	\end{equation*}
	
	Dado un problema $P$ y un punto $x^*$ que es solución factible de $P$, se puede determinar si esta punto cumple las condiciones de KKT. Para ello se implement\'o el siguiente algoritmo:
	
	\begin{figure}[H]
		\begin{minted}[tabsize=4]{python}
def kkt_condition(v_size, g_f, g_h, g_g, x, ineq_constraints):

	active_index_gradients = active_indexes(x, ineq_constraints, g_g)
	
	matrix = []
	for grad in [*g_h, *active_index_gradients]:
		row = [grad(x)[i] for i in range(len(x))]
		matrix.append(row)
		
	lambdas = sp.symbols("l0:%d" % len(g_h))
	mius = sp.symbols("m0:%d" % len(active_index_gradients))
		
	linear_system = equation_system(x, v_size, lambdas, mius, g_f, g_h,
									 active_index_gradients)
	
	solutions = sp.solve(linear_system, [*lambdas, *mius])
	mius_sol = solutions[len(lambdas) :]
	
	if len(solutions) != 0:
		if all(i >= 0 for i in mius_sol):
			return 1
		else:
			return 0
	
	return -1  
		\end{minted}
	\end{figure}

El método recibe la dimensión en la que se está trabajando, los gradientes de las funciones $f, g, h$, el punto que se desea evaluar y las restricciones de las desigualdades, en este mismo orden.

Se determinan las funciones $g_j$ tal que que $j\in I(x^*)$ y se conforma la matriz que está compuesta por los gradientes de las funciones $h_i$ y los $g_j$ activados. Se declaran las variables $\lambda_i$ y $\mu_j$ para conformar el sistema lineal a resolver.

Para solucionar dicho sistema se hizo uso de la biblioteca de Python SymPy, que tiene como propósito reunir todas las caracterísitcas de un sistema de álgebra computacional; es fácilmente extensible y permite la generación de código sencillo y legible.

Una vez procesado el sistema, de acuerdo al resultado arrojado por el algoritmo se determina si el sistema fue:
\begin{itemize}
	\item  compatible determinado, o sea, que tiene una única solución, con lo cual el método retorna 1
	\item compatible indeterminado, que tiene infinitas soluciones, lo cual hace que el método retorne 0
	\item incompatible, sería que no tiene solución, por tanto retorna 0
\end{itemize}

Si se desea acceder al código completo, se puede hacer a través de este \href{utils.py}{enlace}
	
En todos los casos se obtuvo que el sistema era incompatible y por tanto, no se cumplen las condiciones de KKT para los puntos dados. Luego o bien los puntos dados no eran m\'inimos locales del problema o no eran puntos regulares (no cumplen con la LICQ).

Veamos, analíticamente, como se llega a estos resultados. Tomemos como ejemplo, el ejercicio 1a con el punto $x^* = (1.5, 2.5) = (\frac{3}{2}, \frac{5}{2})$
\begin{equation*}
		min \ 3x_1^2 - x_0^2
\end{equation*}

\begin{align*}
	2x_0 - x_1^2 &\leq 0\\
	x_0 &\geq 0\\
	x_1 &\geq 0
\end{align*}

Luego:
\begin{align*}
	f(x) = 3x_1^2 - x_0^2 &\Rightarrow \nabla f(x) = (-2x_0, 6x_1)\\
	g_1(x) = 2x_0 - x_1^2 \leq 0 &\Rightarrow \nabla g_1(x) = (2, -2x_1)\\
	g_2(x) = -x_0 \leq 0 &\Rightarrow \nabla g_2(x) = (-1, 0)\\
	g_3(x) =  -x_1 \leq 0 &\Rightarrow \nabla g_3(x) = (0, -1)
\end{align*}

Para verificar cuales son los \'indices activos de las funciones $g_j$ se eval\'ua la funci\'on $g_j(x^*) = 0$ y si la igualdad se cumple entonces $j \in I(x^*)$
\begin{align*}
	g_1(x^*) &= 2x_0 - x_1^2 = 2 * \frac{3}{2} - (\frac{5}{2})^2 = 3 - \frac{25}{4} \neq 0\\
	g_2(x^*) &= -x_0 = - \frac{3}{2} \neq 0\\
	g_3(x^*) &= -x_1 = - \frac{5}{2} \neq 0
\end{align*}

Por tanto $I(x^*)  = \emptyset$  y para verificar el cumplimiento de la KKt en este punto, basta comprobar la igualdad $\nabla f(x^*) = 0$. 
\begin{equation*}
	\nabla f(x^*) = (-2 * \frac{3}{2}, 6 * \frac{5}{2}) = (-3, 15) \neq 0
\end{equation*}
De donde se puede afirmar que no se cumple la condición de KKT para este punto.

Aplicando de forma an\'aloga este procedimiento en el resto de los ejercicios, se pueden comprobar los resultados obtenidos en el algoritmo presentado.





\end{document}