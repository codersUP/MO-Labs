\documentclass[titlepage, 11pt]{scrartcl}
\usepackage{graphicx}
\usepackage[utf8]{inputenc}
\usepackage{hyperref}
\usepackage{amsmath, amsthm, amsfonts}
\usepackage{graphics}
\usepackage{skull}
\usepackage{mathabx}
\usepackage{float}
\usepackage{epsfig}
\usepackage{amssymb}
\usepackage{dsfont}
\usepackage{latexsym}
\usepackage{newlfont}
\usepackage{epstopdf}
\usepackage{amsthm}
\usepackage{epsfig}
\usepackage{caption}
\usepackage{multirow}
\usepackage{graphics}
\usepackage{wrapfig}
\usepackage[rflt]{floatflt}
\usepackage{multicol}
\usepackage{minted}

\hypersetup{colorlinks,%
	citecolor=black,%
	filecolor=cyan,%
	linkcolor=blue!60,%
	urlcolor=cyan}

\title{
    \normalfont\normalsize
    {\huge Modelos de Optimización\\
    		\textbf{Laboratorio 3}}
    \vspace{12pt}
}

\author{Osmany P\'erez Rodr\'iguez\\
		Enrique Mart\'inez Gonz\'alez\\
		Carmen Irene Cabrera Rodr\'iguez\\
		\textbf{Grupo C412}}

\date{}

\begin{document}
    \maketitle 
    
	\textbf{Condiciones necesarias de Karush-Kuhn-Tucker (KKT) para problemas con restricciones de igualdad y desigualdad.}
	
	Sea $X$ un conjunto abierto no vacío en $R^n$ y sean $f$, $g_j$ y $h_i$ con $i = 1 \ldots m$ y $j = 1\ldots k$, funciones de $R^n$ en $R$. Considere el problema $P: min{f(x) | x \in X, g_j(x) \leq 0, h_i(x) = 0}$. Sea $x^*$ una solución factible del problema $P$. Suponga además que en el punto $x^*$ las funciones $f$, $h_i , \ i = 1 \ldots m$ y $g_j, \ j \in I(x^*)$ son continuamente diferenciables y la función $g_j, \ j\notin I(x^*)$ es continua.
	
	Si $x^*$ es un punto regular y un mínimo local del problema $P$, entonces existen escalares únicos $\lambda_i, \ i = 1 \ldots m$ y $\mu_j, \ j \in I(x^*)$, tal que:
	\begin{equation*}
		\nabla f(x^*) + \sum_{i = 1}^{m}\lambda_i \nabla h_i(x^*) + \sum_{j \in I(x^*)}\mu_j \nabla g_j(x^*) = 0
	\end{equation*}
	\begin{equation*}
		\mu_j \geq 0, \ j \in I(x^*)
	\end{equation*}
	
	\vspace{5mm}
	Para que $x^*$ cumpla la condici\'on de regularidad tiene que ocurrir que los vectores $\nabla g_j(x^*), \ j \in I(x^*)$ y $h_i(x^*), \ i = 1 \ldots m$ sean linealmente independientes ($li$). Esta condici\'on se conoce como LICQ (\textit{Linear Independence Constraint Qualification}).
	\vspace{1cm}
	\section{Algoritmo implementado}
	
	Dado un problema $P$ y un punto $x^*$ que es solución factible de $P$, se puede determinar si este punto cumple las condiciones de KKT. Para ello se implement\'o el siguiente algoritmo (\ref{fig:code}):
	
	El método recibe como parámetros:
	\begin{itemize}
		\item $v\_size$: la dimensión en la que se está trabajando
		\item $g\_f, g\_h, g\_g$: los gradientes de las funciones $f, g, h$ respectivamente
		\item $x$: el punto que se desea evaluar
		\item $ineq\_constraints$: las restricciones de las desigualdades
	\end{itemize}
	
	Se determinan las funciones $g_j$ tal que que $j\in I(x^*)$ y se conforma la matriz que está compuesta por los gradientes de las funciones $h_i$ y los $g_j$ activados. Se declaran las variables $\lambda_i$ y $\mu_j$ para conformar el sistema lineal a resolver.
	
	Para solucionar dicho sistema se hizo uso de la biblioteca de Python \textbf{SymPy}, que tiene como propósito reunir todas las características de un sistema de álgebra computacional; es fácilmente extensible y permite la generación de código sencillo y legible.
	
	Una vez procesado el sistema, de acuerdo al resultado arrojado por el algoritmo se determina si el sistema fue:
	\begin{itemize}
		\item  compatible determinado, o sea, que tiene una única solución, con lo cual el método retorna \textbf{True}
		\item incompatible, sería que no tiene solución, por tanto retorna \textbf{False}
		\item compatible indeterminado, que tiene infinitas soluciones. En este caso, el método retorna la solución obtenida, donde habrán variables libres.
		
	\end{itemize}
	\vspace{7mm}
	\textbf{C\'odigo:}
	\begin{figure}[H]
		\begin{minted}[tabsize=4]{python}
def kkt_condition(v_size, g_f, g_h, g_g, x, ineq_constraints):

	active_index_gradients = active_indexes(x, ineq_constraints, g_g)
	
	matrix = []
	for grad in [*g_h, *active_index_gradients]:
		row = [grad(x)[i] for i in range(len(x))]
		matrix.append(row)
		
	lambdas = sp.symbols("l0:%d" % len(g_h))
	mius = sp.symbols("m0:%d" % len(active_index_gradients))
		
	linear_system = equation_system(x, v_size, lambdas, mius, g_f, g_h,
									 active_index_gradients)
	
	solutions = sp.solve(linear_system, [*lambdas, *mius])

	try:
		mius_sol = solutions[len(lambdas) :]
	except TypeError:
		return solutions
	
	return len(solutions) != 0 and all(i >= 0 for i in mius_sol) 
		\end{minted}
		\caption{Se puede acceder al código completo a través de este \href{utils.py}{enlace}}
		\label{fig:code}
	\end{figure}

	
	
	\vspace{1cm}
	\section{Resultados obtenidos}

\begin{enumerate}
	\item 
	\begin{enumerate}
		\item \begin{equation*}
		min \ 3x_1^2 - x_0^2
		\end{equation*}
		
		\begin{align*}
		2x_0 - x_1^2 &\leq 0\\
		x_0 &\geq 0\\
		x_1 &\geq 0
		\end{align*}
		
		Luego:
		\begin{align*}
		f(x) = 3x_1^2 - x_0^2 &\Rightarrow \nabla f(x) = (-2x_0, 6x_1)\\
		g_1(x) = 2x_0 - x_1^2 \leq 0 &\Rightarrow \nabla g_1(x) = (2, -2x_1)\\
		g_2(x) = -x_0 \leq 0 &\Rightarrow \nabla g_2(x) = (-1, 0)\\
		g_3(x) =  -x_1 \leq 0 &\Rightarrow \nabla g_3(x) = (0, -1)
		\end{align*}
		
		\vspace{7mm}
		\begin{enumerate}
			\item \textbf{Punto} $x^* = (1.5, 2,5) = (\frac{3}{2}, \frac{5}{2})$
			
			El resultado obtenido en este punto fue \textbf{False}, lo que indica que no se cumple la KKT.
			
			Para determinar cuáles son los \'indices activos de las funciones $g_j$, se eval\'ua  $g_j(x^*) = 0$ y si la igualdad se cumple entonces $j \in I(x^*)$
			\begin{align*}
			g_1(x^*) &= 2x_0 - x_1^2 = 2 * \frac{3}{2} - (\frac{5}{2})^2 = 3 - \frac{25}{4} \neq 0\\
			g_2(x^*) &= -x_0 = - \frac{3}{2} \neq 0\\
			g_3(x^*) &= -x_1 = - \frac{5}{2} \neq 0
			\end{align*}
			
			Por tanto $I(x^*)  = \emptyset$, lo cual implica que se cumple la condici\'on de regularidad de $x^*$, o sea, que cumple con la LICQ.
			
			El hecho de que no se cumpla la condición de KKT permite afirmar que o bien el punto no es un m\'inimo local del problema o que no se cumple la LICQ (que no se cumplan ninguna de las dos también podría es una opci\'on); en este caso, s\'i se cumple la condici\'on de regularidad, por tanto el punto $x^*$ no es un m\'inimo local del problema.
			
			\vspace{7mm}
			\item \textbf{Punto} $x^* = (0, 1)$
			
			El resultado obtenido en este punto fue \textbf{False}, lo que indica que no se cumple la KKT.
			
			Determinaci\'on del conjunto de \'indices activos:
			\begin{align*}
			g_1(x^*) &= 2x_0 - x_1^2 = 2 * 0 - 1^2 = -1 \neq 0\\
			g_2(x^*) &= -x_0 = 0\\
			g_3(x^*) &= -x_1 = -1 \neq 0
			\end{align*}
			
			En este caso $I(x^*)  = {2}$; como solo se tiene un vector se cumple la condici\'on de regularidad de $x^*$, o sea, que cumple con la LICQ.
			
			El hecho de que no se cumpla la condición de KKT permite afirmar que o bien el punto no es un m\'inimo local del problema o que no se cumple la LICQ (que no se cumplan ninguna de las dos también podría es una opci\'on); en este caso, s\'i se cumple la condici\'on de regularidad, por tanto el punto $x^*$ no es un m\'inimo local del problema.
			
			\vspace{7mm}
			\item \textbf{Punto} $x^* = (2.75, 3.25) = (\frac{11}{4}, \frac{13}{4})$
			
			El resultado obtenido en este punto fue \textbf{False}, lo que indica que no se cumple la KKT.
			
			Determinaci\'on del conjunto de \'indices activos:
			\begin{align*}
			g_1(x^*) &= 2x_0 - x_1^2 = 2 * \frac{11}{4} - (\frac{13}{4})^2 = \neq 0\\
			g_2(x^*) &= -x_0 = -\frac{11}{4} \neq 0\\
			g_3(x^*) &= -x_1 = -\frac{13}{4} \neq 0
			\end{align*}
			
			En este caso $I(x^*)  = \emptyset$, por tanto se cumple la condici\'on de regularidad de $x^*$, o sea, que cumple con la LICQ.
			
			El hecho de que no se cumpla la condición de KKT permite afirmar que o bien el punto no es un m\'inimo local del problema o que no se cumple la LICQ (que no se cumplan ninguna de las dos también podría es una opci\'on); en este caso, s\'i se cumple la condici\'on de regularidad, por tanto el punto $x^*$ no es un m\'inimo local del problema.
		\end{enumerate}
	
	\vspace{1cm}
		\item \begin{equation*}
		min \ 100(x_1 - x_0^2)^2 + (1 - x_0)^2
		\end{equation*}
		
		\begin{align*}
		x_0^2 - 2x_1 &\leq 5\\
		x_0^2 - x_1 &\leq 100\\
		5x_0^2 - 2 x_1^2 &\leq 500\\
		x_0 &\geq 0\\
		x_1 &\geq 0
		\end{align*}
		
		Luego:
		\begin{align*}
		f(x) = 100(x_1 - x_0^2)^2 + (1 - x_0)^2 &\Rightarrow \nabla f(x) = (400 x_0^3 - 400 x_1x_0 + 2x_0 - 2, 200x_1 - 200x_0^2)\\
		g_1(x) = x_0^2 - 2x_1 - 5 \leq 0 &\Rightarrow \nabla g_1(x) = (2 x_0, -2)\\
		g_2(x) = x_0^2 - x_1 - 100 \leq 0 &\Rightarrow \nabla g_2(x) = (2x_0, -1)\\
		g_3(x) = 5x_0^2 - 2 x_1^2 - 500 \leq 0 &\Rightarrow \nabla g_3(x) = (10x_0, -4x_1) \\
		g_4(x) = -x_0 \leq 0 &\Rightarrow \nabla g_2(x) = (-1, 0)\\
		g_5(x) =  -x_1 \leq 0 &\Rightarrow \nabla g_3(x) = (0, -1)
		\end{align*}
			
			
		\vspace{7mm}
		\begin{enumerate}
			\item \textbf{Punto} $x^* = (0.7, 3.8) = (\frac{7}{10}, \frac{19}{5})$
			
			El resultado obtenido en este punto fue \textbf{False}, lo que indica que no se cumple la KKT.
			
			Determinaci\'on del conjunto de \'indices activos:
			\begin{align*}
				g_1(x^*) &= x_0^2 - 2x_1 - 5 = (\frac{7}{10})^2 - 2* \frac{19}{5} - 5 \neq 0\\
				g_2(x^*) &= x_0^2 - x_1 - 100 = (\frac{7}{10})^2 - \frac{19}{5} - 100 \neq 0\\
				g_3(x^*) &= 5x_0^2 - 2 x_1^2 - 500 = 5 (\frac{7}{10})^2 - 2(\frac{19}{5})^2 - 500 \neq 0\\
				g_4(x^*) &= -x_0 = -\frac{7}{10} \neq 0\\
				g_5(x^*) &= -x_1 = -\frac{19}{5} \neq 0
			\end{align*}
			
			En este caso $I(x^*)  = \emptyset$, por tanto se cumple la condici\'on de regularidad de $x^*$, o sea, que cumple con la LICQ.
			
			El hecho de que no se cumpla la condición de KKT permite afirmar que o bien el punto no es un m\'inimo local del problema o que no se cumple la LICQ (que no se cumplan ninguna de las dos también podría es una opci\'on); en este caso, s\'i se cumple la condici\'on de regularidad, por tanto el punto $x^*$ no es un m\'inimo local del problema.
			
			\item \textbf{Punto} $x^* = (2, 3.5) = (2, \frac{7}{2})$
			
			El resultado obtenido en este punto fue \textbf{False}, lo que indica que no se cumple la KKT.
			
			Determinaci\'on del conjunto de \'indices activos:
			\begin{align*}
			g_1(x^*) &= x_0^2 - 2x_1 - 5 = (2)^2 - 2* \frac{7}{2} - 5 \neq 0\\
			g_2(x^*) &= x_0^2 - x_1 - 100 = 2^2 - \frac{7}{2} - 100 \neq 0\\
			g_3(x^*) &= 5x_0^2 - 2 x_1^2 - 500 = 5 * 2^2 - 2(\frac{7}{2})^2 - 500 \neq 0\\
			g_4(x^*) &= -x_0 = -2 \neq 0\\
			g_5(x^*) &= -x_1 = -\frac{7}{2} \neq 0
			\end{align*}
			
			En este caso $I(x^*)  = \emptyset$, por tanto se cumple la condici\'on de regularidad de $x^*$, o sea, que cumple con la LICQ.
			
			El hecho de que no se cumpla la condición de KKT permite afirmar que o bien el punto no es un m\'inimo local del problema o que no se cumple la LICQ (que no se cumplan ninguna de las dos también podría es una opci\'on); en este caso, s\'i se cumple la condici\'on de regularidad, por tanto el punto $x^*$ no es un m\'inimo local del problema.
			
			\item \textbf{Punto} $x^* = (1.25, 2.75) = (\frac{5}{4}, \frac{11}{4})$

			El resultado obtenido en este punto fue \textbf{False}, lo que indica que no se cumple la KKT.
			
			Determinaci\'on del conjunto de \'indices activos:
			\begin{align*}
			g_1(x^*) &= x_0^2 - 2x_1 - 5 = (\frac{5}{4})^2 - 2* \frac{11}{4} - 5 \neq 0\\
			g_2(x^*) &= x_0^2 - x_1 - 100 = (\frac{5}{4})^2 - \frac{11}{4} - 100 \neq 0\\
			g_3(x^*) &= 5x_0^2 - 2 x_1^2 - 500 = 5 (\frac{5}{4})^2 - 2(\frac{11}{4})^2 - 500 \neq 0\\
			g_4(x^*) &= -x_0 = -\frac{5}{4} \neq 0\\
			g_5(x^*) &= -x_1 = -\frac{11}{4} \neq 0
			\end{align*}
			
			En este caso $I(x^*)  = \emptyset$, por tanto se cumple la condici\'on de regularidad de $x^*$, o sea, que cumple con la LICQ.
			
			El hecho de que no se cumpla la condición de KKT permite afirmar que o bien el punto no es un m\'inimo local del problema o que no se cumple la LICQ (que no se cumplan ninguna de las dos también podría es una opci\'on); en este caso, s\'i se cumple la condici\'on de regularidad, por tanto el punto $x^*$ no es un m\'inimo local del problema.
		\end{enumerate}
	
		\vspace{1cm}
	\item \begin{equation*}
	min \ -x_1^3 - x_0^3
	\end{equation*}
	
	\begin{align*}
	2x_0 - x_1^2 &\leq 70\\
	x_0^2 + x_1^2 &\leq 1000\\
	x_0^2 - 2 x_1^3 &\leq 280\\
	x_0 &\geq 0\\
	x_1 &\geq 0
	\end{align*}
	
	Luego:
	\begin{align*}
	f(x) = -x_1^3 - x_0^3 &\Rightarrow \nabla f(x) = (-3x_0^2, -3x_1^2)\\
	g_1(x) = 2x_0 - x_1^2 - 70\leq 0 &\Rightarrow \nabla g_1(x) = (2, -2x_1)\\
	g_2(x) = x_0^2 + x_1^2- 1000 \leq 0 &\Rightarrow \nabla g_2(x) = (2x_0, 2x_1)\\
	g_3(x) = x_0^2 - 2 x_1^3 - 280 \leq 0 &\Rightarrow \nabla g_3(x) = (2x_0, -6x_1^2) \\
	g_4(x) = -x_0 \leq 0 &\Rightarrow \nabla g_2(x) = (-1, 0)\\
	g_5(x) =  -x_1 \leq 0 &\Rightarrow \nabla g_3(x) = (0, -1)
	\end{align*}
	
		\begin{enumerate}
			\item \textbf{Punto} $x^* = (0.8, 2.7) = (\frac{8}{10}, \frac{27}{10})$
			
			El resultado obtenido en este punto fue \textbf{False}, lo que indica que no se cumple la KKT.
			
			Determinaci\'on del conjunto de \'indices activos:
			\begin{align*}
			g_1(x^*) &= 2x_0 - x_1^2 - 70 = 2 * \frac{8}{10} - (\frac{27}{10})^2 - 70 = -75.69\neq 0\\
			g_2(x^*) &= x_0^2 + x_1^2- 1000 = (\frac{8}{10})^2 + (\frac{27}{10})^2 - 1000 = -992.07 \neq 0\\
			g_3(x^*) &= x_0^2 - 2 x_1^3 - 280 = (\frac{8}{10})^2 - 2 (\frac{27}{10})^3 - 280 = -318.726 \neq 0\\
			g_4(x^*) &= -x_0 = -\frac{8}{10} \neq 0\\
			g_5(x^*) &= -x_1 = -\frac{27}{10} \neq 0
			\end{align*}
			
			En este caso $I(x^*)  = \emptyset$, por tanto se cumple la condici\'on de regularidad de $x^*$, o sea, que cumple con la LICQ.
			
			El hecho de que no se cumpla la condición de KKT permite afirmar que o bien el punto no es un m\'inimo local del problema o que no se cumple la LICQ (que no se cumplan ninguna de las dos también podría es una opci\'on); en este caso, s\'i se cumple la condici\'on de regularidad, por tanto el punto $x^*$ no es un m\'inimo local del problema.
			
			
		\vspace{7mm}
		\item \textbf{Punto} $x^* = (5.6, 7.1)$
	
		El resultado obtenido en este punto fue \textbf{False}, lo que indica que no se cumple la KKT.
		
		Determinaci\'on del conjunto de \'indices activos:
		\begin{align*}
		g_1(x^*) &= 2x_0 - x_1^2 - 70 = 2 * 5.6 - (7.1)^2 - 70 = -109.21\neq 0\\
		g_2(x^*) &= x_0^2 + x_1^2- 1000 = (5.6)^2 + (7.1)^2 -1000 = -918.23 \neq 0\\
		g_3(x^*) &= x_0^2 - 2 x_1^3 - 280 = (5.6)^2 - 2(7.1)^3 - 280 = -964.46 \neq 0\\
		g_4(x^*) &= -x_0 = -5.6 \neq 0\\
		g_5(x^*) &= -x_1 = -7.1 \neq 0
		\end{align*}
		
		En este caso $I(x^*)  = \emptyset$, por tanto se cumple la condici\'on de regularidad de $x^*$, o sea, que cumple con la LICQ.
		
		El hecho de que no se cumpla la condición de KKT permite afirmar que o bien el punto no es un m\'inimo local del problema o que no se cumple la LICQ (que no se cumplan ninguna de las dos también podría es una opci\'on); en este caso, s\'i se cumple la condici\'on de regularidad, por tanto el punto $x^*$ no es un m\'inimo local del problema.
		
		\vspace{7mm}
		\item \textbf{Punto} $x^* = (3.4, 4.9)$
		
		El resultado obtenido en este punto fue \textbf{False}, lo que indica que no se cumple la KKT.
		
		Determinaci\'on del conjunto de \'indices activos:
		\begin{align*}
		g_1(x^*) &= 2x_0 - x_1^2 - 70 = 2 * 3.4 - (4.9)^2 - 70 = -87.21\neq 0\\
		g_2(x^*) &= x_0^2 + x_1^2- 1000 = (3.4)^2 + (4.9)^2 -1000 = -964.43 \neq 0\\
		g_3(x^*) &= x_0^2 - 2 x_1^3 - 280 = (3.4)^2 - 2(4.9)^3 - 280 = -33.14 \neq 0\\
		g_4(x^*) &= -x_0 = -3.4\neq 0\\
		g_5(x^*) &= -x_1 = -4.9\neq 0
		\end{align*}
		
		En este caso $I(x^*)  = \emptyset$, por tanto se cumple la condici\'on de regularidad de $x^*$, o sea, que cumple con la LICQ.
		
		El hecho de que no se cumpla la condición de KKT permite afirmar que o bien el punto no es un m\'inimo local del problema o que no se cumple la LICQ (que no se cumplan ninguna de las dos también podría es una opci\'on); en este caso, s\'i se cumple la condici\'on de regularidad, por tanto el punto $x^*$ no es un m\'inimo local del problema.
		\end{enumerate}

		
	\end{enumerate}
\end{enumerate}

\vspace{1cm}
Suponiendo que dado un punto, este cumpla las condiciones de KKT, esto no significa necesariamente que el sea un mínimo y por tanto, soluci\'on del problema. Si adem\'as de que cumpla la KKT se tiene que la funci\'on objetivo y las funciones de restricci\'on $g_j$ son convexas, y $h_i$ son funciones afines; esto s\'i constituye condici\'on suficiente para que el punto mencionado sea un m\'inimo global. En el caso de los problemas vistos anteriormente; no se tienen restricciones de igualdad, por tanto bastar\'ia confirmar la convexidad de las funciones $f$ y $g_j$.





\end{document}